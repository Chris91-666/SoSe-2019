\documentclass[12pt,fleqn]{report}

\setcounter{tocdepth}{3}
\setcounter{secnumdepth}{3}

\usepackage{a4}
\usepackage{latexsym}
\usepackage{epsf}
\usepackage{listings}
\usepackage{xcolor}
\usepackage{titlesec}
\usepackage{lipsum}
\usepackage{scrextend}
\usepackage{tikz}
\usetikzlibrary{fit}
\usetikzlibrary{calc}
\usetikzlibrary{positioning}
\usetikzlibrary{shadows}
\usetikzlibrary{shadows.blur}
\usetikzlibrary{shapes.symbols}
\usepackage{fontspec}
\usepackage{wrapfig}
\usepackage{nameref}
\usepackage{hyperref}
\usepackage{enumitem}
\usepackage{relsize}
\usepackage{pdfpages}
\usepackage{makeidx}

\usepackage{amsmath, amsthm, amssymb}
\usepackage[ngerman, german]{babel}
\usepackage{marvosym}
\usepackage{graphics}
\usepackage{extarrows}

\usepackage{hyperref}% http://ctan.org/pkg/hyperref
\usepackage{cleveref}% http://ctan.org/pkg/cleveref
\usepackage{lipsum}% http://ctan.org/pkg/lipsum
\newtheorem{theorem}{Theorem}
\newtheorem{lemma}{Lemma}
\crefname{lemma}{Lemma}{Lemmas}

\newenvironment{beweis}{\begin{proof}[Beweis]}{\end{proof}}


\usepackage[font=footnotesize,labelfont=bf]{caption}

\newfontfamily\sansfont[Extension = .ttf, Path = ./fonts/, UprightFont = OpenSans-Regular, BoldFont=OpenSans-SemiBold, ItalicFont=OpenSans-SemiBold]{OpenSans}


\titleformat{\chapter}[display]
  {\sansfont\bfseries}{}{0pt}{\huge}
  
\renewcommand{\contentsname}{\sansfont Content}  

\definecolor{folderbg}{RGB}{70, 130, 180}
\definecolor{folderborder}{RGB}{50,50,50}
\newlength\Size
\setlength\Size{5pt}


\pgfdeclarelayer{bg}
\pgfsetlayers{bg,main}

\renewcommand{\textfraction}{0.05}
\topmargin -15mm
\textwidth 160mm
\textheight 240mm
\oddsidemargin -2mm
\evensidemargin -2mm

\newcommand\blfootnote[1]{%
  \begingroup
  \renewcommand\thefootnote{}\footnote{#1}%
  \addtocounter{footnote}{-1}%
  \endgroup
}


\usepackage[backend=biber,bibencoding=utf8,style=numeric,autocite=plain,sorting=none]{biblatex}
\usepackage{setspace}

\addbibresource{Literatur.bib}

\titleformat*{\section}{\Large\sansfont}
\titleformat*{\subsection}{\large\sansfont}
\titleformat*{\subsubsection}{\sansfont}

\begin{document}
\pagenumbering{roman}
\begin{titlepage}
\title{\vspace{-5cm}\small Saarland University\\
Faculty of Natural Sciences and Technology I\\
Department of Mathematic\\
\mbox{} \vspace{4cm}\\
\small\sansfont Mitschrift \\
\mbox{} \vspace{0.2cm}\\
  {\Huge\sansfont  Stochastik 1}\\}



\author{\textbf{\sansfont \scriptsize gehalten von}\\
Prof. Dr. Christian Bender\\
\scriptsize\sansfont\textbf{Sommersemester 2019}\\
\vspace{1cm}\\
%\textbf{\sansfont Reviewers:}\\
%Prof. Dr. Markus Bläser\\
\vspace{1cm}\\
}
\date{}
\end{titlepage}
\maketitle


\pagebreak
{\sansfont \small
\tableofcontents
}
\cleardoublepage
\pagenumbering{arabic}




\part{Grundlagen}

\chapter{Einführung und Notationen}

\section{Zufallsexperimente}

Unter Zufallsexperimenten verstehen wir Experimente, deren Ausgänge zufälligen Einflüssen unterliegen, z.B.:
\begin{itemize}
\item[•] das Würfeln mit einem 6-seitigen Würfel
\item[•] das Drehen eines Glücksrades
\item[•] der Verlauf eines Aktienkurses (im kommenden Jahr)
\end{itemize}
\underline{Frage:}\newline
Wie kann man derartige Zufallsexperimente mathematisch modellieren?\newline
Es bezeichne $\Omega$ eine nicht-leere Menge, die alle möglichen Ergebnisse des Zufallsexperimentes umfasst. $\Omega$ wird \textbf{Stichprobenraum}, \textbf{Ergebnismenge} oder \textbf{Grundraum} genannt.\newline
Die Elemente $\omega \in \Omega$ heißen \textbf{Ergebnisse}.\newline
\newline
\underline{Beispiele:}
\begin{itemize}
\item[•] Beim Würfeln kann man $\Omega = \{1, 2, 3, 4, 5, 6\}$ wählen.
\item[•] Beschreibt man die Position des Glücksrades durch den Winkel zur x-Achse, so bietet sich hier $\Omega = (0,2\pi]$
\item[•] Der Aktienverlauf im kommenden Jahr kann als Funktion von $[0,1]$ nach $\mathbb{R}$ aufgefasst werden, sodass hier gilt:\newline 
			$\Omega = \mathbb{R}^{[0,1]} := \{\omega \mid \omega : [0,1] \rightarrow \mathbb{R}\}$
\end{itemize}
Teilmengen $a \subset \Omega$ nennen wir \textbf{Ereignisse}.
Zum Beispiel beschreibt beim Würfeln mit einem echten Würfel (6 seitig und fair) $A = \{1, 3, 5\}$ das Ereignis "Es fällt eine ungerade Zahl".
Wir sagen, ein Ereignis $A$ tritt ein, falls bei einem Zufallsexperiment ein $\omega \in A$ realisiert wird.
Wichtige Ereignisse sind:
\begin{itemize}
\item $\emptyset$ : unmögliches Ereignis
\item $\Omega$ : sicheres Ereignis
\item $\{\omega\}$, $\omega \in \Omega$ : Elementarereignis
\end{itemize}




\end{document}
